\chapter{Análise e Discussão dos Resultados}

Após o fim do estudo, iniciou-se a análise das respostas obtidas nos questionários aplicados. Todas as perguntas eram opcionais pelo intuito de garantir que apenas respostas com qualidade fossem informadas.

\section{How do you choose between different Ruby Gems}

Com relação às duas primeiras perguntas do questionário, ou seja, ao conhecimento de desenvolvimento de \textit{software} e em específico o desenvolvimento de \textit{software} em Ruby, todos os participantes desenvolviam software em algum nível seja este nível profissional, não-profissional ou contribui para \textit{software} livre. Das 668 respostas, com relação ao tempo de uso de Ruby os participantes se subdividiam em:

\begin{enumerate}
	\item 4 não responderam
	\item 26 dos participantes não utilizavam Ruby
    \item 119 dos participantes utilizavam Ruby a menos de um ano
    \item 201 dos participantes utilizavam Ruby entre um a dois anos
    \item 177 dos participantes utilizavam Ruby entre três a cinco anos
    \item 119 mais de cinco anos
\end{enumerate}

Com relação aos fatores decisivos para a escolha entre duas gems os participantes citaram diferentes abordagens, porém as palavras-chaves mais mencionadas foram:

\begin{enumerate}
	\item Conceitos mais mencionados
	    \begin{enumerate}
	        \item \textbf{Popularidade}: 114 citações
			\item \textbf{Documentação}: 193 citações    	
    		\item \textbf{Atividade no Repositório}: 222 citações
        \end{enumerate}
   	\item Sites mais mencionados
      	\begin{enumerate}
        	\item \textbf{Rubygems}: 91 citações
            \item \textbf{Stack Overflow}: 106 citações
		    \item \textbf{Google}: 125 citações
		    \item \textbf{Ruby Toolbox}: 172 citações
            \item \textbf{GitHub}: 239 citações
	    \end{enumerate}
\end{enumerate}

Quando perguntados se existia espaço para melhoramentos no quesito de busca de Gems, a maioria das respostas se posicionaram de forma negativa, porém com diferentes posturas. Por exemplo:

\begin{enumerate}
	\item rubygems.org does a pretty good job with this, at least for my needs.
    \item I like of the https://www.ruby-toolbox.com/.
    \item ruby-toolbox.com is pretty good. If there were something that rated documentation and test coverage, those could be useful.
    \item the great tool called Google combines a lot of these websites together.
    \item ruby-toolbox.com, it's the best this kind of thing now i think
    \item I'm pretty happy with Ruby Toolbox.
\end{enumerate}

Aqueles questionados que tiveram posturas positivas geralmente também sugeriam funcionalidades adicionais. Algumas dessas funcionalidades encaixavam exatamente com a ferramenta proposta. Por exemplo:

\begin{enumerate}
	\item A tool where developers can give a score to the gems for several points like: code style, code coverage, code architecture, and show the results ordered by any of this ones.
    \item it would be nice to have reviews on a gem like Amazon does for products.
    \item I would like something that either categorized them by function or showed most used more accurately
    \item It's nice to know how active the gem is/how up to date it is.  What issues are people having with it
    \item There should be webcrawlers rating the Gems around the web using a page-rank algorithm to give insigths on trends, heated discussions, problems on open fórums like StackOverflow, and etc. Like eBay for Gems, where I can use advanced filters to find them.
    \item I think that ruby gems.org could be improved to a more social network style, with comments, discussions and examples
Searching a Gems for it's name is useless.
\end{enumerate}

\section{GitHub API}

O comparativo entre os resultados obtidos está apresentado separado por categorias no Apêndice B. É visível a discrepância das indicações do Ruby Toolbox, com relação ao real uso das Gems, em todas as categorias estudadas exceto nas categorias de \textit{Frameworks} de teste e Construtores de Formulários.

\section{Desenvolvimento e Análise da Ferramenta}

Cada participante e Gem foi anonimizado. Toda referência a um participante será feita na forma de P\# onde \# se refere ao identificador do participante. Toda referência a uma Gem será feita na forma de G\# onde \# se refere ao identificador do participante.

Com relação à familiarização dos questionados às Gems, 57\% deles eram familiarizados com ambas as Gems da discussão, 28\% deles eram familiarizados com uma das duas e 15\% deles não eram familiarizados com nenhuma delas.

Com relação ao conhecimento das Gems, 100\% dos participantes já utilizaram G1 e apenas P2 e P7 tinham utilizados ambas Gems.

Após a leitura da discussão ocorreu mudança de opinião apenas nos participantes P1 e P3. Em particular P3 atestou que a ferramenta lhe proporcionou mais conhecimentos. P3 comentou o seguinte: 'Sobre a gem que havia utilizado, nada mudou. Porém, o outro participante ofereceu algumas idéias interessantes sobre o uso da outra que ainda não havia pensado ou lido sobre.'.

Após o uso da ferramenta 72\% dos questionados afirmaram que, dependendo do contexto, ambas Gems podem ser utilizadas.

Com relação à análise dos participantes em relação à ferramenta:
\begin{enumerate}
  \item \textbf{P1:} Achei a ferramente interessante. Seria legal que houvesse um meio de votar em cada um dos argumentos, assim teríamos uma forma de feedback instantâneo sobre a veracidade do argumento, visto que os participantes podem falar o que quiserem (sendo ou não verdade).
  \item \textbf{P2:} A ferramenta é bem interessante, uma vez que possibilita uma discussão interessante sobre dois tópicos, facilitando a vida de pessoas em busca de qual ferramenta escolher
  \item \textbf{P3:}Os participantes poderiam dar uma espécie de "like" para indicar que concorda com o argumento do outro. Poderia haver um campo final para "consenso" onde os dois participantes poderiam escrever um resumo da conversa
  \item \textbf{P5:} Achei prático e interessante. 
  \item \textbf{P6:} É uma ferramenta interessante, a comparação entre ferramentas, frameworks, etc. é um assunto sempre recorrente na área de software. Acredito que um ponto interessante seria destacar melhor os pontos positivos e negativos que cada pessoa possui sobre os objetos de comparação.
  \item \textbf{P7:} Ficou interessante, dá pra se abordar vários prós e contras de diversas ferramentas.
\end{enumerate}

P4 não respondeu a esta pergunta.