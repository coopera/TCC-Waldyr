\chapter{Referencial Teórico}

\section{A busca por conhecimento do engenheiro de software}

Todas atividades do processo de desenvolvimento de \textit{software} requer constantes atualizações dos desenvolvedores, devido ao grande fluxo de inovações que surgem ininterruptamente. \cite{Singer2014}.

As redes sociais é uma das variadas formas utilizadas pelos desenvolvedores para se atualizar \cite{Treude2012} \cite{Storey:2014:ESM:2593882.2593887}. Outra fonte de conhecimento encontra-se em sua equipe de trabalho, haja vista a frequente busca em obter ajuda dos seus colegas \cite{Weinberg1998}.

Dentre os variados motivos da busca por conhecimento podemos citar, por exemplo, a evolução do \textit{software}, a qualidade do \textit{software} e a proliferação de ferramentas e ambientes de desenvolvimento de \textit{software} \cite{Jazayeri:2004:ESE:1025115.1025201}. Em particular o último motivo pode ser apresentado pelas fontes de forma divergente. Por exemplo, quando a equipe de trabalho e as redes sociais indicam ferramentas ou ambientes divergentes para um mesmo problema ou um conjunto de problemas relacionados.

\section{Gestão da informação} 

A gestão de conhecimento tem como metas a aquisição de novos conhecimentos, a manutenção do conhecimento existente para garantir seu uso futuro, permeando o seu armazenamento e sua difusão, bem como a sua reprodução em novos contextos \cite{Bjornson2008}.

A qualidade na gestão da informação permite às organizações de desenvolvimento de \textit{software} se manter competitivas \cite{Rabelo2015}.

Dessa forma, é possível ver organizações de \textit{software} padronizando e compartilhando seus ambientes e ferramentas prepostos com suas respectivas equipes de desenvolvimento.

\section{GitHub}

GitHub\footnote{\url{http://github.com}} é um serviço web de repositórios de código utilizando primariamente Git como controle de versão~\cite{Figueira2015}. Atualmente possui cerca de onze milhões e meio de desenvolvedores e vinte e oito milhões de repositórios cadastrados\footnote{\url{https://github.com/about/press}}.

O que destaca o GitHub das demais ferramentas de repositório \textit{online} é o grande apelo deste para a disponibilidade de projetos de código aberto.

É comum que empresas de desenvolvimento de software criem organizações dentro do GitHub. Organizações funcionam como um agregado de usuários com acesso comum aos mesmos repositórios. Cada usuário continua mantendo uma conta própria para si.

Por ser o maior centro de hospedagem de código da atualidade \cite{Gousios2012}, se mostra uma excelente plataforma para se obter exemplos de código de diversas linguagens e finalidades.

Esta mesma plataforma de desenvolvimento é extensivamente estudado. Como por exemplo os efeitos das interações sociais dentro do GitHub\cite{Syeed:2014:SCR:2641580.2641586}. Também podemos citar as análises de Kabbedijk e Jansen \cite{conf/icsob/KabbedijkJ11} no repositório Git de Ruby, resultando na definição de três grandes papéis através da interação de Gems e desenvolvedores. Dabbish et al. \cite{dabbish2012social} analizou a rede social do Github e seus resultados dissertam sobre o visível apoio que o Github provê à colaboração e aprendizado. Robbes et al. \cite{robbes2011study} por outro lado, estudou a evolução de sistemas de software livre baseado no Efeito Ripple. Este estudo, mais especificamente, partem dos fundamentos de alguns dos resultados de Margaret-Anne, et al \cite{Storey:2014:ESM:2593882.2593887}. Desde os primeiros dias da engenharia de software, os desenvolvedores têm usado, inovado, adaptado e adotado as mídias para incrementar as suas interações sociais com outros desenvolvedores.

\section{Ruby, Gems e RubyGems}

\subsection{Ruby}

Ruby é uma linguagem de programação dinâmica com uma complexa, porém expressiva, gramática e uma biblioteca de classes central com uma rica e poderosa API. Ruby toma inspirações das linguagens Lisp, Smalltalk e Perl, apesar de fazer uso de uma gramática mais facilmente compreensível para programadores C e Java.

Ruby é uma linguagem puramente orientada a objetos, entretanto também é adequada para estilos de programações variados, tais como, por exemplo, procedural e funcional. Possui poderosas capacidades de meta programação e pode ser usada para criar uma linguagem de domínio específico, também conhecidas como DSLs.

Esses e outros aspectos são fatores importantes na construção e manutenção de sistemas robustos e livres de complexidades tanto no desenvolvimento quanto no uso, tornando-os amigáveis para os desenvolvedores, em aspectos estruturais. \cite{flanagan2008ruby}

\subsection{Gem e RubyGems}

Uma Gem é um módulo, biblioteca ou plugin responsável por encapsular uma solução para um problema que pode-se instalar em um determinado ambiente Ruby. Uma Gem é composta pelos seguintes componentes: O código Ruby, a sua respectiva documentação e um arquivo chamado gemspec. Uma Gem deve seguir a mesma estrutura padronizada de organização de código.

Cada Gem possui um nome, uma versão e uma plataforma. Por exemplo, por exemplo a Gem chamada `\textit{rake}' \cite{rake2014}. Essa Gem existe na versão 0.8.7 (de Maio, 2009); A plataforma utilizada pela Gem é `\textit{ruby}', cuja plataforma é qualquer plataforma que Ruby funcione. \cite{berube2007practical}

Plataformas são baseadas em na arquitetura da unidade central de processamento, tipo de sistema operacional e muitas vezes também são influenciadas pela versão do sistema operacional. Instâncias de plataformas podem ser citadas, tais como, entre outras, `\textit{java}' e `\textit{x86-mingw32}'. As plataformas são responsáveis por indicar o ambiente específico requerido.

Tudo isso é dito no Gemspec. O Gemspec é responsável por guardar os metadados da sua aplicação, tais como, entre outros, a lógica da aplicação, os usuários da Gem e quem é responsável por escrevê-la. Este arquivo é um dos mais importantes pois ele ajudará o Bundler a configurar o ambiente sem muita dificuldade.

Por praticidade, a comunidade de desenvolvimento Ruby criou um servidor de hospedagem de Gems chamado RubyGems tornando mais prático e centralizado a difusão, busca e utilização. Na atualidade, o RubyGems possui 110,789 Gems e 96,113 usuários.
