\chapter{Metodologia}

Este trabalho prevê o desenvolvimento de uma ferramenta para auxiliar discussões sobre tópicos de assuntos variados. Todos os questionários estão presentes no Apêndice A.

\section{Estudo 1: How do you choose between different Ruby Gems}

O primeiro estudo buscou responder as perguntas de pesquisa seguindo os seguintes procedimentos.

\subsection{Coleta de Dados}

Distribuiu-se um questionário para desenvolvedores participantes do GitHub podendo ser de diferentes nacionalidades, frameworks, linguagens e etc. Para torná-lo mais acessível, o questionário foi feito em inglês. A amostra foi selecionada utilizando amostragem de conveniência, ou seja, as amostras foram selecionadas baseadas no julgamento do autor. Foram obtidas 668 respostas para este questionário.

\subsection{Análise de Dados}

As respostas deveriam deixar claro:
\begin{enumerate}
	\item Como os desenvolvedores Ruby escolhiam uma entre duas Gems, quando ambas satisfaziam suas necessidades.
    \item O que se demonstra relevante para o desenvolvedor durante o processo de escolha dessa Gem.
    \item Se existe alguma necessidade, durante o processo, não atendida com as atuais ferramentas.
\end{enumerate}

Após a coleta e análise das respostas, entendeu-se como o processo de escolha ocorria e evidenciou-se os principais fatores responsáveis por ressaltar a qualidade em uma Gem, a ponto de destacar uma Gem em relação a outra neste processo. Além disso, também foi evidenciado as principais ferramentas utilizadas pelos desenvolvedores e suas respectivas falhas. 

Desta forma, respondeu-se todas perguntas de pesquisa, com exceção da quarta pergunta. Logo, caracterizou-se a principal ferramenta como aquela mais citada e realizou-se um novo estudo (Estudo 2) para identificar a extensão do uso de suas indicações e consequentemente responder a quarta e restante pergunta.

Com as necessidades não atendidas e sugestões de melhorias, surgiu-se a ideia para uma ferramenta alternativa.

\section{Estudo 2: Qual extensão do uso da principal ferramenta}

A resposta da quarta pergunta de pesquisa foi elaborada após coleta e mineração de dados através da GitHub API\footnote{\url{https://developer.github.com/v3/}}. 

\subsection{Coleta de Dados}

A principal ferramenta categoriza as Gems em 168 diferentes categorias, tais como, entre outras, Qualidade de Código, Geradores de PDF e CSS. Seguindo esta categorização, 10 categorias foram selecionadas ao acaso e buscou-se informações sobre os projetos usuários.

A fim de minimizar o ruído dos dados minerados e coletar conteúdo com qualidade \cite{Kalliamvakou:2014:PPM:2597073.2597074}, foi levado em consideração apenas repositórios ativos cuja linguagem predominante era Ruby. Além disso, estes projetos constavam em repositórios raízes, ou seja projetos hospedados em repositórios que não são ramificações de outros projetos também hospedados em outros repositórios.

\subsection{Análise de Dados}

Após análise foi identificado duas discrepâncias com relação às indicações da principal ferramenta.

\begin{enumerate}
	\item A quantidade de projetos usuários não corroborava com a indicação
    \item Os projetos usuários que aparentavam ter mais qualidade, baseados nos argumentos dos entrevistados do Estudo 1, preferiam as Gems com menor indicação.
\end{enumerate}

Esse estudo, além de responder a quarta e restante pergunta de pesquisa, fortaleceu a hipótese da necessidade de uma ferramenta alternativa.

\section{Desenvolvimento e Análise da Ferramenta}

\subsection{Desenvolvimento da Ferramenta}

Após ser feito o Estudo 1 e 2, pôde-se concluir que necessitava-se de uma ferramenta alternativa. Os principais requisitos foram levantados

\begin{enumerate}

  \item \textbf{Suporte à Tomada de decisão}. Auxiliar a decisão entre duas ou mais Gems que endereçam o mesmo problema.
  
  \item \textbf{Conteúdo informativo}. Auxiliar a apreensão das dificuldades e facilidades promovidas por cada uma das Gems em discussão.
  
  \item \textbf{Promover Senso Crítico}. Auxiliar o desenvolvimento do senso crítico dos desenvolvedores em relação às opções de soluções e aos problemas que as mesmas endereçam.
  
\end{enumerate}

\section{Estudo 3: Análise da Ferramenta Argue}

O terceiro estudo buscou a validação da ferramenta proposta criando-se um questionário, intitulado: 'Questionário de Avaliação da Ferramenta', e o mesmo apresenta-se no Apêndice A.

Para análise da ferramenta foi instanciado um ambiente com uma real discussão de dois reais desenvolvedores entre duas Gems, Ransack\footnote{\url{https://github.com/activerecord-hackery/ransack/}} e ElasticSearch\footnote{\url{https://github.com/elastic/elasticsearch-ruby}} cujas soluções se endereçam aos mesmos problemas, ambas auxiliam a criação de buscas.

\subsection{Coleta de Dados}

O público alvo deste questionário foi limitado e aplicado apenas em alunos da Empresa Júnior de Engenharia de Software da Universidade Federal do Rio Grande do Norte, que já possuíam alguma experiência profissional em Ruby e Rails e que fazem parte da comunidade de Ruby local. Foram obtidas 7 respostas para este questionário.

\subsection{Análise de Dados}

O questionário preocupa-se em relatar o que ocorreu com a opinião de cada um dos questionados sobre as Gems em questão após a utilização da ferramenta. Especificamente, como a ferramenta deve ser mais informativa do que indicativa, procurou observar se ocorreram variações nos posicionamentos dos questionados para um posicionamento onde uma futura utilização dependeriam apenas do contexto e da necessidade do problema que elas iriam resolver.