\chapter{Metodologia}

Durante o processo de desenvolvimento deste trabalho foram aplicados dois questionários; Uma mineração de dados do Github para colher informações sobre de Gems específicas e confrontar os resultados com os índices do RubyToolbox; Foi desenvolvido e proposto uma ferramenta para auxiliar discussões sobre tópicos de assuntos variados e, finalmente, um novo questionário para avaliar se a ferramenta pode auxiliar pessoas nas suas tomadas de decisões.

O primeiro questionário desenvolvido e aplicado, a partir de agora este questionário será referenciado de Questionário 1, foi o questionário intitulado 'How do you choose between different Ruby Gems'. Seu objetivo era entender como os desenvolvedores Ruby escolhiam entre duas Gems, quando ambas satisfaziam suas necessidades. Também buscamos obter respostas para o que se demonstrava relevante para o desenvolvedor durante o processo de escolha dessa Gem além de perguntar se alguma necessidade durante o processo não era atendida com as atuais ferramentas, sendo elas qualquer uma das ferramentas que auxilie o processo, tais como, entre outras, Google, Stack Overflow e RubyToolbox.

Dessa forma, buscáva entender fatores responsáveis por evidenciar a qualidade em uma Gem, a ponto de destacar uma Gem em relação a outra no processo de escolha entre Gems, cujas soluções enderessavam os mesmos problemas. Além disso, também buscava entender e mostrar quais os principais locais em que os desenvolvedores procuravam informações e demonstrações sobre Gems. Em particular, verificamos se seria mencionado o RubyToolbox como local de busca ou se seu índice era um fator de decisão na escolha de Gems.

Suportando a parte específica do primeiro questionário, dados foram coletados e minerados do Github através da Github API. Os dados coletados se resumem aos projetos que constam em repositórios raízes, ou seja projetos hospedados em repositórios que não são ramificações de outros projetos também hospedados em repositórios, que incluem uma determinada Gem em seu Gemspec, também chamada da especificação da Gem. Dessa forma poderíamos verificar se realmente os projetos seguiam ou pelo menos eram influênciados pelas indicações do RubyToolbox.

Nesse meio tempo uma ferramenta e um questionário foi desenvolvidos. A ferramenta responsável por auxiliar a desição entre duas Gems que endereçam o mesmo problema. Apesar de ter características semelhantes com o RubyToolbox a abordagem é mais informacional e deixa a cargo do usuário de entender os pros e contras das Gems em questão e utilizá-las de acordo com o seu entendimento. O questionário foi feito para suportar a ferramenta e garantir que o seu foco e objetivo sejam os esclarecidos anteriormente.