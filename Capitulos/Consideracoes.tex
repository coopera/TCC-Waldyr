\chapter{Considerações Finais}

\section{Lições aprendidas}

Ao longo do desenvolvimento do estudo, foi evidenciado que os desenvolvedores costumam buscar conhecimento e soluções em diversos lugares, corroborando com \cite{parnin2012crowd}. Em particular, no ambiente do estudo, Rubygems, Stack Overflow, Google, Ruby Toolbox e GitHub são os locais mais acessados respectivamente.

Também foi possível evidenciar que os desenvolvedores associam a qualidade de um projeto à qualidade de componentes deste projeto, como por exemplo: \textit{Stars}, \textit{forks}, qualidade de documentação, facilidade de manutenção, facilidade de utilização, fechamento de \textit{issues}, data dos ultimos \textit{commit} e etc.

\section{Benefícios dos Resultados}

Os resultados dos estudos indicam o uso extensivo de motores de busca, como por exemplo os índices de popularidade do GitHub para escolher entre duas Gems que enderessam os mesmos problemas. Entretanto o \textit{ranking} de Gems gerado pelo serviço Ruby Toolbox, baseado nos índices do GitHub, não é levado em conta na decisão.

A ferramenta implementada no estudo, podendo ser considerada uma versão simplificada do Discourse\footnote{\url{https://www.discourse.org/}}, atua como suporte fazendo com que o desenvolvedor aprenda sobre os pros e contras de cada uma das Gems e deixe a decisão a cargo do próprio desenvolvedor.

\section{Estudos Futuros}

Estudos futuros podem validar a aplicabilidade da ferramenta, seja ela o Argue ou Discourse, em contextos mais próximos da realidade. Por exemplo, utilizar um maior espaço amostral ou fazer o mesmo estudo, porém em um ambiente empresarial.